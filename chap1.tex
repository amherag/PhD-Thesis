\chapter{Introduction}
\label{chapter:introduction}

% Reading this section, I think this is actually a better Introducion, start
% with this section. Start by presenting the Problem.

There are systems where the number of elements that conform it is considerably
low or it is easy to understand how the elements interact among them. For
example, analyzing projectile trajectories is a rather easy to understand
problem, where the number of significant factors involved in the problem is
low. As another example, we can perform a structural analysis of a building, and
even though the number of loads being involved in the analysis can be very high,
one can still understand how the loads are going to interact among them with a
very high precision. In contrast, there are systems where the elements that are
interacting in it are very hard to understand or the number of elements is
incredibly high. For example, understanding why human beings can change their
religious beliefs is a situation which is very hard to understand, or to
calculate what is going to be the final position of thousands of paper sheets
thrown in the air at the same time is a task which is almost impossible to
do. These systems that are hard to model by simple mathematical formulas are
known as complex systems.

A financial market is an example of a complex system, where thousands of traders
are constantly deciding how many units are going to be sold or bought at
arbitrary intervals of time. The strategies followed by each of these traders
can vary from technical analysis using mathematical formulas to basing their
decision on news interpretations. And even after establishing a trading
strategy, traders can decide to ignore their conclusions due to subjective
factors such as "having a bad feeling about this." As a consequence, financial
markets are frequently used to showcase complexity theory concepts
\ref{Arthur1999} \ref{Bundesbank2007}. Even though these markets can exhibit an
almost chaotic behavior, certain patterns still emerge, which stop financial
markets from being labeled as chaotic systems \ref{Castillo2001}.

% Add references in: % "financial markets are frequently used to showcase
% complexity theory concepts" % which stop financial markets from being labeled
% as chaotic systems.

One of the methods used for forecasting how a financial market will behave in
price is to use fundamental analysis \cite{Kadiri2015}. This type of analysis
takes into account information that you can gather from the factors that
influence the market's price. For example, if a company releases their financial
statements and they demonstrate that the company is performing financially well,
it is very likely that a surge in the company's stock price will occur. In this
case, a number of traders will become aware of this information and will decide
to buy the company's stock.

In contrast to fundamental analysis, technical analysis can be used to forecast
the behavior of a market \cite{Kadiri2015}. Technical analysis is conformed by
any process that forecasts price movements where only data related to past and
current prices of a financial market is involved. For example, a traditional
technique is to use moving averages to obtain a price which is representative of
the past $N$ prices \cite{Li1999}. In technical analysis it is common to chart
these methods -- which are called technical indicators -- along with the real
prices of a financial market, as can be seen in Figure
\ref{figure:moving-average-example}. This way a trader can see all the
information in a graphical manner, and have everything plotted on a single
chart.

% complex systems -> financial markets -> fundamental analysis -> technical
% analysis -> fibonacci retracements -> people have different
% interpretations -> people don't think in a crisp way and the last two
% are the problem: different interpretations, many people,
% and how do we represent this thought

\begin{figure}
  \caption{Moving average example} \centering
  \includegraphics[width=1.0\textwidth]{img/moving-average.png}
  \label{figure:moving-average-example}
\end{figure}

Some researchers believe that technical analysis works because it falls under
the category of a self-fulfilling prophecy: it works because many people use it
\cite{salganik2008leading}. If enough people buy an asset because technical
analysis predicts that it is a good idea to do so, the price will rise and then
technical analysis is correct. Nevertheless, there is a big number of technical
indicators, and many of these technical indicators are parametrized, which means
that there are many possible ways of interpreting a market using technical
indicators. Still, there is enough empirical evidence that demonstrates that
technical analysis can help a trader to better understand a market
\cite{Maranon2018}. A possible explanation behind this is that, although there
are virtually an infinite number of possible configurations that a technical
indicator can adopt, not all of them seem to "follow the market," i.e. a
technical indicator needs to be configured to match the behavior of past prices,
and not simply use an arbitrary configuration. This would further reinforce the
self-fulfilling prophecy hypothesis behind technical analysis: the traders that
are creating the price movements in a market are using technical indicators that
are being adjusted to the price that they created themselves.

Taking into consideration the discussion in the previous paragraph, technical
indicators thus work by trying to find a pattern in the price movements. These
patterns are then interpreted in some way by traders, and inferences about the
market are created, such as: "this market is presenting a high volatility" or
"this market is going uptrend." This also implies that technical indicators can
be interpreted in different ways, depending on the trader's knowledge and
opinion.

In conclusion, in order to understand a market from a technical analysis point
of view, one needs to take into consideration that a number of traders with
different ideas and different interpretations of the market are creating the
price movements. Forecasting or modelling tools taking a technical analysis
approach should take into consideration these factors.

\section{Thesis Structure}
\label{section:thesis-structure}

% Comments are placed after each sentence % consider doing a hard wrap to 80 or
% so characters so.  % This will make easier to place comments on each line, for
% GitHub

The objective of this thesis is to propose a novel series of approaches on how
to analize and interpret a financial market to obtain insights that can be
used to describe the behavior behind that market and create a profitable trading
strategy. In this first Chapter the reader finds the motivation of this work,
such as why understanding a financial market is a problem and why solving this
problem is beneficial (see Section \ref{section:justification}).

% In the first sentence "propose a novel series of approaches to how a financial
% market can be interpreted and analyzed"


% the things you approach "interpreted and analyzed" are to far away, maybe:
% ... approaches on how to analize and interpret a financial market ...  % we
% must try not to use words such as "great importance" change to? " %
% ... solving this problem is beneficial".

The proposed method in this document is based on at least three different
fields, which are computational intelligence, complexity theory and financial
markets. Chapter \ref{chapter:preliminaries} is dedicated to explaining the
concepts related to these fields, which are needed to understand the contents of
the following Chapters and the thesis overall.

% Depending on space the comment on the background of readers could be deleted.

Once the reader is acquainted with the necessary terminology, concepts and ideas
discussed in the aforementioned Chapter, a series of related works is presented
in Chapter \ref{chapter:related-work}. These works help the reader to further
understand the problem being solved in this thesis, as one can see how different
approaches have been taken in the past to solve similar problems. Also, the
presented related works indirectly fuel the justification of this thesis, as the
reader can realize how similar approaches to the proposed method have improved
other fields of research.

The actual proposed method of this thesis is presented in Chapter
\ref{chapter:proposed-method}. In this Chapter the reader can find how the
problem discussed in Section \ref{section:description-of-the-problem} can be
solved. Additionally, some other proposals are discussed, such as how
intuitionistic fuzzy sets and systems can be used to represent a knowledge in an
expanded way, in contrast to traditional fuzzy sets and systems.

A particular implementation of the proposed method is discussed in Chapter
\ref{chapter:implementation}, which covers some technical details, such as what
programming languages were used and how exactly the ideas from Chapter
\ref{chapter:proposed-method} were programmed. As each of the concepts discussed
in Chapter \ref{chapter:proposed-method} are covered in the implementation, some
of the Sections from both Chapters match in name.

After having implemented the proposed method, a series of experiments were
performed to demonstrate how it compares to other similar solutions. The
experiments also help the reader to understand how the method can be used to
obtain insights from a financial market, as well as how one can use these
insights to perform better decisions while trading such markets. These
experiments are presented in Chapter \ref{chapter:experiments}, where their
design is discussed, and the results obtained from the experiments are presented
in Chapter \ref{chapter:results}.

Lastly, the reader can find some conclusions about the work developed for this
thesis in Chapter \ref{chapter:conclusions}, as well as various propositions on
how the present work could be further developed in the future in Chapter
\ref{chapter:future-work}.

\section{Justification}
\label{section:justification}

Understanding the nature of financial markets is of utter importance, as
financial markets play a big role in the economy system of any country. Having
several tools to understand financial markets can help to forecast financial
crises, understand the development of the economy of a company, and take better
decisions when investing.

The problem stated in Section \ref{section:description-of-the-problem} can be
broken into the following pieces: a forecasting or modelling solution needs to
consider 1) a number of traders manipulating the price movement in a financial
market, 2) each of these traders have different opinions about how to trade a
market depending on how the market is behaving, and 3) each of these traders
interpret the market differently.

Traders can be seen as agents in a multi-agent system, where each of these
agents receive data from another agent: a broker. The broker is in charge of
sending the price information about a financial market to every trader. This
data can be used later as the input for all the technical indicators. The agents
representing the traders never interact among them, only with the broker. In
addition to receiving the financial market data from the broker, a trader can
also send transaction operations to the broker, such as buy or sell orders. Many
authors consider multi-agent systems and agent-based models as one of the better
tools for modelling and simulating financial markets \cite{Lebaron2001}
\cite{Gamil2007} \cite{Boer-Sorban2008}, due to their flexibility to represent
very complex systems.

Implementing an agent-based solution allows not only to simulate the actions
that different entities can take in a complex system, but also to model how the
agents are perceiving their environment and the inference process used to take
such actions. This is interesting because the simulated models created using an
agent-based solution can then be examined by looking at how each of the agents
behaves in the system.

%% Then talk about how to model the agents' actions or belfies, etc.

%% We can justify two things: why the problem is important and why the
% method works

%% why multi-agent, why intuitionistic, why retracements
