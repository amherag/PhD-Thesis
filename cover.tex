% -*-latex-*-
% 
% For questions, comments, concerns or complaints:
% thesis@mit.edu
% 
%
% $Log: cover.tex,v $
% Revision 1.8  2008/05/13 15:02:15  jdreed
% Degree month is June, not May.  Added note about prevdegrees.
% Arthur Smith's title updated
%
% Revision 1.7  2001/02/08 18:53:16  boojum
% changed some \newpages to \cleardoublepages
%
% Revision 1.6  1999/10/21 14:49:31  boojum
% changed comment referring to documentstyle
%
% Revision 1.5  1999/10/21 14:39:04  boojum
% *** empty log message ***
%
% Revision 1.4  1997/04/18  17:54:10  othomas
% added page numbers on abstract and cover, and made 1 abstract
% page the default rather than 2.  (anne hunter tells me this
% is the new institute standard.)
%
% Revision 1.4  1997/04/18  17:54:10  othomas
% added page numbers on abstract and cover, and made 1 abstract
% page the default rather than 2.  (anne hunter tells me this
% is the new institute standard.)
%
% Revision 1.3  93/05/17  17:06:29  starflt
% Added acknowledgements section (suggested by tompalka)
% 
% Revision 1.2  92/04/22  13:13:13  epeisach
% Fixes for 1991 course 6 requirements
% Phrase "and to grant others the right to do so" has been added to 
% permission clause
% Second copy of abstract is not counted as separate pages so numbering works
% out
% 
% Revision 1.1  92/04/22  13:08:20  epeisach

% NOTE:
% These templates make an effort to conform to the MIT Thesis specifications,
% however the specifications can change.  We recommend that you verify the
% layout of your title page with your thesis advisor and/or the MIT 
% Libraries before printing your final copy.

%% \title{Clasificaci??n de Estados Afectivos usando Din??mica de Tecleo yde Rat??n en Programaci??n de Software}

%% \author{Amaury Hern??ndez ??guila}
%% % If you wish to list your previous degrees on the cover page, use the 
%% % previous degrees command:
%% %       \prevdegrees{A.A., Harvard University (1985)}
%% % You can use the \\ command to list multiple previous degrees
%% %       \prevdegrees{B.S., University of California (1978) \\
%% %                    S.M., Massachusetts Institute of Technology (1981)}
%% \department{Divisi??n de Estudios de Posgrado e Investigaci??n}

%% % If the thesis is for two degrees simultaneously, list them both
%% % separated by \and like this:
%% % \degree{Doctor of Philosophy \and Master of Science}
%% \degree{Maestro en Ciencias Computacionales}

%% % As of the 2007-08 academic year, valid degree months are September, 
%% % February, or June.  The default is June.
%% \degreemonth{Julio}
%% \degreeyear{2014}
%% \thesisdate{8 de septiembre, 2014}
%% \copyrightnoticetext{Tijuana, Baja California, M??xico}

%% %% By default, the thesis will be copyrighted to MIT.  If you need to copyright
%% %% the thesis to yourself, just specify the `vi' documentclass option.  If for
%% %% some reason you want to exactly specify the copyright notice text, you can
%% %% use the \copyrightnoticetext command.  
%% %\copyrightnoticetext{\copyright IBM, 1990.  Do not open till Xmas.}

%% % If there is more than one supervisor, use the \supervisor command
%% % once for each.
%% %\supervisor{Dr. Jos?? Mario Garc??a Valdez}{.}

%% % This is the department committee chairman, not the thesis committee
%% % chairman.  You should replace this with your Department's Committee
%% % Chairman.
%% \chairman{M.C. Alejandra Mancilla Soto}{.}

% Make the titlepage based on the above information.  If you need
% something special and can't use the standard form, you can specify
% the exact text of the titlepage yourself.  Put it in a titlepage
% environment and leave blank lines where you want vertical space.
% The spaces will be adjusted to fill the entire page.  The dotted
% lines for the signatures are made with the \signature command.
%\maketitle

\begin{titlepage}
  \begin{center}
  \textsc{\Large SEP}
  \:\:\:\:\:\:\:\:\:\:\:\:\:\:\:\:\:\:\:\:\:\:\:\:\:\:\:\:\:\:\:\:\:\:\:\:
\:\:\:\:\:\:\:\:\:\:\:\:\:\:\:\:\:\:\:\:\:\:\:\:\:\:\:\:\:\:\:\:\:\:\:\:
\:\:\:\:\:\:\:\:\:\:\:\:\:\:\:\:
  \textsc{\Large TNM}~\\[1cm]
  \textsc{\Large INSTITUTO TECNÓLOGICO DE TIJUANA}~\\[1cm]
  \textsc{\Large División de Estudios de Posgrado e
    Investigación}~\\[0.5cm]

    \includegraphics[width=0.25\textwidth]{./logos.png}~\\[1cm]

    %% \textsc{\LARGE Trading Strategy for Foreign Exchange Markets based on Computational Intelligence, Multi-Agent Systems and Logarithmic Spirals}~\\[1.5cm]
    
    \textsc{\LARGE Estrategia de Intercambio para
Mercados Financieros basada en
Computación Inteligente, Sistemas
Multi-Agente y Retrocesos}~\\[1.5cm]

%% \textsc{por}~\\[0.5cm]
%% \textsc{\Large Amaury Hernández Águila}~\\[1.0cm]

%% \textsc{Tesis para obtener el grado de}~\\[0.5cm]
%% \textsc{\Large Doctor en Ciencias Computacionales}~\\[0.5cm]
%% \textsc{}~\\[0.5cm]

%% \textsc{Febrero 2019}~\\[0.2cm]
%% \textsc{Tijuana, Baja California, México}~\\[1cm]

\begin{minipage}{1\textwidth}
  \begin{flushright} \large
    TRABAJO DE TESIS~\\[0.5cm]
    \small
    PRESENTADO POR~\\[0.5cm]
    \large
    AMAURY HERNÁNDEZ ÁGUILA~\\[0.5cm]
    \small
    PARA OBTENER EL GRADO DE~\\[0.5cm]
    \large
    DOCTOR EN CIENCIAS~\\
    EN COMPUTACIÓN~\\[0.5cm]
    \small
    DIRECTOR DE TESIS~\\
    \large
    DR. JOSÉ MARIO GARCÍA VALDEZ~\\[0.5cm]
    \small
    Tijuana, B.C., México. Mayo 2019
  \end{flushright}
\end{minipage}

%% \begin{minipage}{1.5\textwidth}
%%   \begin{flushleft} \large
%%     \emph{Director:}\\
%%     Dr. Jos?? Mario \textsc{Garc??a Valdez}
%%   \end{flushleft}
%% \end{minipage}

%% \begin{minipage}{1.5\textwidth}
%% \begin{flushleft} \large
%% \emph{Co-Directora:} \\
%% M.C. Alejandra \textsc{Mancilla Soto}
%% \end{flushleft}
%% \end{minipage}

\vfill

  \end{center}

\end{titlepage}

% The abstractpage environment sets up everything on the page except
% the text itself.  The title and other header material are put at the
% top of the page, and the supervisors are listed at the bottom.  A
% new page is begun both before and after.  Of course, an abstract may
% be more than one page itself.  If you need more control over the
% format of the page, you can use the abstract environment, which puts
% the word "Abstract" at the beginning and single spaces its text.

%% You can either \input (*not* \include) your abstract file, or you can put
%% the text of the abstract directly between the \begin{abstractpage} and
%% \end{abstractpage} commands.

% First copy: start a new page, and save the page number.
%\cleardoublepage
% Uncomment the next line if you do NOT want a page number on your
% abstract and acknowledgments pages.
\pagestyle{empty}
\setcounter{savepage}{\thepage}
\begin{abstractpage}
% $Log: abstract.tex,v $
% Revision 1.1  93/05/14  14:56:25  starflt
% Initial revision
% 
% Revision 1.1  90/05/04  10:41:01  lwvanels
% Initial revision
% 
%
%% The text of your abstract and nothing else (other than comments) goes here.
%% It will be single-spaced and the rest of the text that is supposed to go on
%% the abstract page will be generated by the abstractpage environment.  This
%% file should be \input (not \include 'd) from cover.tex.

El entender el comportamiento de los mercados financieros es importante para determinar el estado de la economía de un país. Las herramientas actuales para entender estos comportamientos son variadas y usualmente pueden ser asignadas a una de dos categorías: análisis fundamentalista o técnico. El análisis fundamentalista se basa en analizar cualquier factor que pueda afectar el valor de un activo, tales como condiciones financieras o la administración de una compañía, mientras que el análisis técnico se enfoca exclusivamente en analizar los movimientos de los precios de un mercado. Esta tesis presenta un método de predicción para mercados financieros basado en análisis técnico. El método puede ser usado como una estrategia de intercambio y como una herramienta para entender el comportamiento de un mercado financiero y sigue una arquitectura basada en agentes donde las reglas de los agentes están basadas en lógica difusa. Específicamente se usan sistemas difusos intuicionistas que le permiten a los agentes modelar tanto la incertidumbre como la indecisión en un mercado. Para poder modelar la percepción de los agentes, los precios de un mercado son preprocesados usando un algoritmo basado en un indicador técnico basado en retrasos. Una implementación del método propuesto demuestra la versatilidad del sistema, ya que los modelos generados pueden llevar a estrategias de intercambio rentables, así como modelos que pueden ser interpretados por un ser humano para obtener posibles explicaciones sobre el comportamiento de un mercado.

\clearpage
\section*{Abstract}

Understanding financial markets is paramount in order to determine the well-being of a country's economy. The currently available tools to obtain such understandings are varied and can usually fall under one of two categories: fundamental or technical analysis. Fundamental analysis relies on analyzing any factor that can affect an asset's value, such as financial conditions and company management, whereas technical analysis relies exclusively on analyzing a market price movements. This thesis presents a financial market forecasting method based on technical analysis. The method can be used as a trading strategy and as a tool for understanding the behavior of a financial market and follows an agent-based architecture where the agent rules are based on fuzzy logic. Specifically, intuitionistic fuzzy systems are used to allow the agents model the uncertainty and hesitancy in a market. In order to model the agents' perception, the market price data is preprocessed using an algorithm based on a retracements technical indicator. An implementation of the proposed method shows the versatility of the system, as the generated models can provide profitable trading strategies, as well as models that can be interpreted by a human being to obtain possible explanations for a market's behavior.
\end{abstractpage}

% Additional copy: start a new page, and reset the page number.  This way,
% the second copy of the abstract is not counted as separate pages.
% Uncomment the next 6 lines if you need two copies of the abstract
% page.
% \setcounter{page}{\thesavepage}
% \begin{abstractpage}
% % $Log: abstract.tex,v $
% Revision 1.1  93/05/14  14:56:25  starflt
% Initial revision
% 
% Revision 1.1  90/05/04  10:41:01  lwvanels
% Initial revision
% 
%
%% The text of your abstract and nothing else (other than comments) goes here.
%% It will be single-spaced and the rest of the text that is supposed to go on
%% the abstract page will be generated by the abstractpage environment.  This
%% file should be \input (not \include 'd) from cover.tex.

El entender el comportamiento de los mercados financieros es importante para determinar el estado de la economía de un país. Las herramientas actuales para entender estos comportamientos son variadas y usualmente pueden ser asignadas a una de dos categorías: análisis fundamentalista o técnico. El análisis fundamentalista se basa en analizar cualquier factor que pueda afectar el valor de un activo, tales como condiciones financieras o la administración de una compañía, mientras que el análisis técnico se enfoca exclusivamente en analizar los movimientos de los precios de un mercado. Esta tesis presenta un método de predicción para mercados financieros basado en análisis técnico. El método puede ser usado como una estrategia de intercambio y como una herramienta para entender el comportamiento de un mercado financiero y sigue una arquitectura basada en agentes donde las reglas de los agentes están basadas en lógica difusa. Específicamente se usan sistemas difusos intuicionistas que le permiten a los agentes modelar tanto la incertidumbre como la indecisión en un mercado. Para poder modelar la percepción de los agentes, los precios de un mercado son preprocesados usando un algoritmo basado en un indicador técnico basado en retrasos. Una implementación del método propuesto demuestra la versatilidad del sistema, ya que los modelos generados pueden llevar a estrategias de intercambio rentables, así como modelos que pueden ser interpretados por un ser humano para obtener posibles explicaciones sobre el comportamiento de un mercado.

\clearpage
\section*{Abstract}

Understanding financial markets is paramount in order to determine the well-being of a country's economy. The currently available tools to obtain such understandings are varied and can usually fall under one of two categories: fundamental or technical analysis. Fundamental analysis relies on analyzing any factor that can affect an asset's value, such as financial conditions and company management, whereas technical analysis relies exclusively on analyzing a market price movements. This thesis presents a financial market forecasting method based on technical analysis. The method can be used as a trading strategy and as a tool for understanding the behavior of a financial market and follows an agent-based architecture where the agent rules are based on fuzzy logic. Specifically, intuitionistic fuzzy systems are used to allow the agents model the uncertainty and hesitancy in a market. In order to model the agents' perception, the market price data is preprocessed using an algorithm based on a retracements technical indicator. An implementation of the proposed method shows the versatility of the system, as the generated models can provide profitable trading strategies, as well as models that can be interpreted by a human being to obtain possible explanations for a market's behavior.
% \end{abstractpage}

%\cleardoublepage
\clearpage

\section*{Agradecimientos}

Indiscutiblemente la persona a la que debo agradecer más por haberme motivado a concluír mi doctorado es a mi esposa Susana. Sin su apoyo emocional jamás habría concluído siquiera el primer semestre.

Gracias a mi familia, la cual jugó un papel importantísimo en mi desarrollo académico, ya que siempre me ha exigido dar lo mejor de mí y sin lugar a duda lo mejor de mí requiere obtener este grado. Gracias en especial a mi papá y a mi mamá por siempre motivarme y apoyarme con mis estudios.

También jamás olvidaré todo el apoyo que me brindaron mis amigos durante este periodo: Martin, Christian, Jonathan y mi director de tesis José Mario.

Por último, pero no en importancia, quiero agradecer al Instituto Tecnológico de Tijuana por haberme dado siempre una educación de alta calidad, así como al CONACYT por haberme otorgado una beca a través del Programa Nacional de Posgrados de Calidad (PNPC), con el número de CVU 479351. De igual forma, quiero agradecer a todos los profesores del posgrado que, sinceramente, me han sorprendido con la calidad de sus conocimientos y enseñanzas. Especialmente quiero agradecer al Doctor Castillo y a la Doctora Melin por cuidar con tanta meticulosidad la calidad del posgrado y, de nuevo, a mi director de tesis por haberme guiado tanto en mis estudios de doctorado como en los de maestría.

%%%%%%%%%%%%%%%%%%%%%%%%%%%%%%%%%%%%%%%%%%%%%%%%%%%%%%%%%%%%%%%%%%%%%%
% -*-latex-*-
